%%%%%%%%%%%%%%%%%%%%%%%%%%%%%%%%%%%%%%%%%%%%%%%%%%%%%%%%%%%%%%%
%
% Welcome to Overleaf --- just edit your LaTeX on the left,
% and we'll compile it for you on the right. If you open the
% 'Share' menu, you can invite other users to edit at the same
% time. See www.overleaf.com/learn for more info. Enjoy!
%
%%%%%%%%%%%%%%%%%%%%%%%%%%%%%%%%%%%%%%%%%%%%%%%%%%%%%%%%%%%%%%%
\documentclass{beamer}
\usepackage{amsmath}
%Information to be included in the title page:
\usefonttheme{serif}
\title{Topics in Algebra, Chapter 1 Solutions}
\author{David Sillman}
\date{2022}

\begin{document}

\frame{\titlepage}

\begin{frame}
\frametitle{1.1 - Set Theory}
\small
\begin{enumerate}
    \item [(1a)]
    If $A\subset B$ and $B\subset C$, then every element $a\in A$ is also $a\in B$. Likewise, $B\subset C \Leftrightarrow \left(a\in B \Rightarrow a\in C\right)$. Thus, every element $a\in A$ is also in $C$. Therefore, $A\subset C$.
    \item [(1b)]
    If we presuppose that $B\subset A$, this means that $x\in B \Rightarrow x\in A$. In set-builder notation, we know that \begin{equation*}
        A\cup B = \left\{ x \mid x\in A \vee x\in B \right\}
    \end{equation*} 
    Because every element of $B$ is also in $A$, the set builder is reduced to 
    \begin{equation*}
        A\cup B = \{x\mid x\in A \} = A
    \end{equation*}

    Conversely, if we start by supposing that $A\cup B = A$, this implies, by logical necessity, that
    \begin{equation*}
        (x\in A) \Leftrightarrow (x\in A \vee x\in B)
    \end{equation*}
    As alluded to above, this is logically transformable into the statement that $x\in B \Rightarrow x\in A$, and thus $B\subset A$.
\end{enumerate}
\end{frame}
\begin{frame}
\frametitle{1.1 - Set Theory}
\small
\begin{enumerate}
    \item [(1c)] \quad In the set $B \cup C$, every element satisfies $x\in B \vee x\in C$. Likewise, in $A \cup C$, every element satisfies $x\in A \vee x\in C$. In the case that $x\in C$, the right side of the disjunction is satisfied in both sets. If $x\in B$, then $B\subset A$ implies that $x\in A$, and so both sets are satisfied again. This means that every element in $B\cup C$ is in $A\cup C$, and so $B\cup C \subset A\cup C$.
    
    \quad Similarly, every element of $B\cap C$ satisfies $x\in B \wedge x\in C$ and every element of $A\cap C$ satisfies $x\in A \wedge x\in C$. Because, again, due to $B\subset A$, we have that $x\in B \Rightarrow x\in A$ and thus every element in $B\cap C$ must also be in $A\cap C$ and so $B\cap C \subset A \cap C$.
\end{enumerate}
\end{frame}
\begin{frame}
\frametitle{1.1 - Set Theory}
\small
\begin{enumerate}
    \item [(2a)] The commutativity of set intersection ($\cap$) and set union ($\cup$) follow from the analogous commutativity properties of logical operators $\wedge$ and $\vee$, respectively. Because $A\cap B$ consists of the elements which satisfy
    \begin{equation*}
    	x\in A \wedge x\in B \quad\equiv\quad x\in B \wedge x\in A
    \end{equation*}
    and thus it follows that $A\cap B = B\cap A$. The same argument follows for $A \cup B = B\cup A$.
    \item [(2b)] As with the last problem, the associativity of $\cap$ follows from the associativity of logical $\wedge$. That is, the set $(A\cap B)\cap C$ consists of elements which satisfy,
    \begin{equation*}
    	(x\in A \wedge x\in B)\wedge x\in C\quad\equiv\quad x\in A\wedge(x\in B \wedge x\in C)
    \end{equation*}
    And thus, exactly those same elements satisfy the necessary condition for being elements of $A\cap(B\cap C)$, giving the equality $(A\cap B)\cap C = A\cap(B\cap C)$.
\end{enumerate}
\end{frame}
\begin{frame}
\frametitle{1.1 - Set Theory}
\small
\begin{enumerate}
	\item[(3)] Logically speaking, the elements of $A\cup(B\cap C)$ are those which satisfy $x\in A \vee (x\in B \wedge x\in C)$. By the distributivity of $\vee$, it follows that this logical condition is equivalent to the condition,
	\begin{equation*}
		x\in A \vee (x\in B \wedge x\in C)\quad\equiv\quad(x\in A\vee x\in B)\wedge (x\in A \vee x\in C)
	\end{equation*}
	And thus, the elements in the former set must be exactly those elements in the latter set, giving the equality $A\cup(B\cap C) = (A\cup B)\cap(A\cup C)$.
	\item[(4a)] The elements of $(A\cap B)'$ are those which satisfy $\neg(x\in A\wedge x\in B)$. Applying De Morgan's logical negation rules, we get that this is logically equivalent to $x\notin A\vee x\notin B$. The set whose elements all satisfy this condition is identically $A'\cup B'$.
	\item[(4b)] As in the last problem, we translate the set $(A\cup B)'$ into the membership condition $\neg(x\in A \vee x\in B)$. De Morgan transforms this into the membership condition $x\notin A \wedge x\notin B$. This membership condition corresponds to the identical set, $A'\cap B'$.
\end{enumerate}
\end{frame}
\begin{frame}
\frametitle{1.1 - Set Theory}
\small
\begin{enumerate}
	\item[(5)] There are two cases: (i) $A$ is disjoint from $B$ ($A\cap B = \emptyset$), or (ii) $A$ is not disjoint from $B$ ($A\cap B \neq \emptyset$).
	
	\quad In case (i), we trivially have that $o(A\cap B) = o(\emptyset) = 0$. So, it is trivially the case that every element of $A$ is represented in $A\cup B$ and likewise for $B$, and all of these elements are distinct. Therefore, $o(A\cup B) = o(A) + o(B)$, which works for our hypothesis given that $o(A\cap B) = 0$.
	
	\quad In case (ii), let's suppose that there are $k$ elements in common between $A$ and $B$, such that $o(A\cap B) = k$. Because $A\cup B$ will only contain one copy of each of the common elements, we must subtract $o(A\cap B)$ from $o(A) + o(B)$ to get the number of unique elements in $A\cup B$.
\end{enumerate}
\end{frame}
\begin{frame}
\frametitle{1.1 - Set Theory}
\small
\begin{enumerate}
	\item[(6)] First, we construct on $A$ a bijective index $i: A\to [n]$. Every subset $S\subset A$ is uniquely identified by an $n$-tuple of binary variables $(b_1,\ldots,b_n)\in B$, with $b_i\in \{0, 1\}$. It's trivial to see that the set of all binary $n$-tuples $B$ has $2^n$ elements. From this set, we have a bijection which generates (and indexes) the set of all subsets of $A$ (hereafter called the \textit{power set} of $A$, $\mathcal{P}(A)$). That is, we define $\sigma: B\to \mathcal{P}(A)$ via the map 
	\begin{equation*}	
	\sigma: b\mapsto \bigcup_{b_j=1} i^{-1}(j)
	\end{equation*}
	In other words, if a particular subset had $b_j = 1$, this means the element $a\in A$ with $i(a) = j$ is included in the subset. Because there exists a bijective map between the finite sets $B$ and $\mathcal{P}(A)$, we have that they are the same size. Therefore, $\mathcal{P}(A)$ has $2^n$ elements.
\end{enumerate}
\end{frame}
\begin{frame}
\frametitle{1.1 - Set Theory}
\small
\begin{enumerate}
	\item[(7)] Let $S$ be the set of Americans. Let $C$ be the set of Americans that like cheese, and $A$ be the set of Americans that like apples. The proportions given suggest that $|C|/|S| = 0.63$ and that $|A|/|S| = 0.76$. Because both $C\subset S$ and $A\subset S$, we expect $C\cup A\subset S$. This necessitates that
	\begin{gather*}
	|C\cup A| \leq |S| \\
	|C| + |A| - |C\cap A| \leq |S|
	\end{gather*}
	Rearranging terms and dividing through by $|S|$, we get a bound on the proportion of Americans which like both cheese and apples:
	\begin{gather*}
	|C\cap A| \geq \frac{|C|}{|S|} + \frac{|A|}{|S|} - 1.0 \\
	|C\cap A| \geq 0.63 + 0.76 - 1.0 \\
	|C\cap A| \geq 0.39
	\end{gather*}
	In English, no fewer than $39\%$ of Americans like both cheese and apples.
\end{enumerate}
\end{frame}
\begin{frame}
\frametitle{1.1 - Set Theory}
\small
\begin{enumerate}
	\item[(8)] Recall that set difference $A - B$ entails the membership condition $x\in A \wedge x\notin B$. This means that the \textit{symmetric difference} $A*B$ entails the membership condition,
	\begin{equation*}
	(x\in A \wedge x\notin B)\vee(x\in B\wedge x\notin A)
	\end{equation*}
	By double-distributing the disjunction, we get to a CNF representation, and then set-difference:
	\begin{gather*}
	(x\in A\vee x\in B)\wedge(x\in A\vee x\notin A)\;\wedge \\ (x\notin B\vee x\in B)\wedge(x\notin B\vee x\notin A) \\
	\Downarrow \\ 
	(x\in A\vee x\in B)\wedge(x\notin B\vee x\notin A) \\
	\Downarrow \\
	(x\in A\vee x\in B)\wedge\neg(x\in A\wedge x\in B)
	\end{gather*}
	This shows logical equivalence in membership between $A*B$ and $(A\cup B) - (A\cap B)$.
\end{enumerate}
\end{frame}
\begin{frame}
\frametitle{1.1 - Set Theory}
\small
\begin{enumerate}
	\item[(9a)] \quad For brevity, we argue non-symbolically. Every element of $(A+B)$ is either unique to $A$ or unique to $B$. Taking the symmetric difference, then, $(A+B)+C$ must result in those elements which are either unique to $A$, unique to $B$, unique to $C$, or common to all three of them. 
	
	\quad Likewise, the elements of $(B+C)$ are those elements unique to $B$ and unique to $C$. Taking the symmetric difference, we see that $A+(B+C)$ is the set of elements which are unique to $A$, unique to $B$, unique to $C$, or common to all three. Therefore, the sets are equal and so $(A+B)+C=A+(B+C)$. 
\end{enumerate}
\end{frame}
\begin{frame}
\frametitle{1.1 - Set Theory}
\small
\begin{enumerate}
	\item[(9b)] Reducing the symmetric differences and distributing the leftmost conjunction,
	\begin{gather*}
	A\cap\left((B-C)\cup(C-B)\right) \\
	\left(A\cap(B - C)\right)\cup\left(A\cap(C - B)\right)
	\end{gather*}
	We then use the fact that intersection distributes over set difference:
	\begin{gather*}
	\left((A\cap B)-(A\cap C)\right)\cup\left((A\cap C) - (A\cap B)\right)\\
	\Downarrow \\
	A\cdot B + A\cdot C
	\end{gather*}
	\item[(9c)] It's trivially the case that $A\cdot A = A\cap A = A$.
\end{enumerate}
\end{frame}
\begin{frame}
\frametitle{1.1 - Set Theory}
\small
\begin{enumerate}
	\item[(9d)] Every element of $A$ is also (identically) an element of $A$, so there are no elements unique to either set. Therefore, $A+A=\emptyset$
	\item[(9e)] Taking the left-symmetric-difference from $A$ of both sides of the equation and applying the associativity from part (a) with the cancellation property of part (d), we get:
	\begin{gather*}
	A+(A+B)=A+(A+C) \\
	(A+A)+B=(A+A)+C) \\
	\emptyset + B = \emptyset + C
	\end{gather*}
	Trivially, the symmetric difference of any set with the empty set is simply the set itself, which gives us $B = C$.
\end{enumerate}
\end{frame}
\begin{frame}
\frametitle{1.1 - Set Theory}
\small
\begin{enumerate}
	\item[(10a)] The relation of having a common ancestor is reflexive, symmetric and transitive. Therefore, it is a valid equivalence relation.
	\item[(10b)] The relation of living within 100 miles of each other is reflexive and symmetric, but not transitive. Therefore, it is not a valid equivalence relation.
	\item[(10c)] The relation of having the same father is reflexive, symmetric and transitive. Therefore, it is a valid equivalence relation.
	\item[(10d)] The relation of having the same absolute value is reflexive, symmetric and transitive. Therefore, it is a valid equivalence relation.
	\item[(10e)] The relation of being strictly greater and strictly lesser than one another is impossible to satisfy. Therefore, it is not a valid equivalence relation.
	\item[(10f)] The relation of two lines having the same slope in the plane is reflexive, symmetric and transitive. Therefore, it is a valid equivalence relation.
\end{enumerate}
\end{frame}
\begin{frame}
\frametitle{1.1 - Set Theory}
\small
\begin{enumerate}
	\item[(11a)] Using only symmetry and transitivity, we do not have a guarantee that there exists a $b$ for $a$ such that $a\sim b$. Thus, this argument does not account for cases that each equivalence class $[a]$ are each only individual elements. 
	\item[(11b)] If we include a property known as \textit{seriality}, which necessitates that every $a$ has a $b$ such that $a\sim b$. Under this assumption, both symmetry and transitivity imply reflexivity.
\end{enumerate}
\end{frame}
\begin{frame}
\frametitle{1.1 - Set Theory}
\small
\begin{enumerate}
	\item[(12)] Clearly, $a\sim a$ because $0$ is a multiple of $n$. Likewise, if $a\sim b$, then it means $a - b = pn$ and so $b - a = -pn$, which is also a multiple of $n$, implying $b \sim a$. Finally, $a\sim b$ and $b\sim c$ mean that $a - b = pn$ and $b - c = qn$. Adding the equations together gives us $a - c = (p + q)n$, so $a\sim c$. Therefore, differing by a multiple of $n$ is a valid equivalence relation. Each of the equivalence classes are defined to be $\text{cl}(i) = \{x\in \mathbb Z \mid x\equiv i\pmod{n}\}$. Because every integer must have a remainder in $[n]$ after division by $n$, we have that $\text{cl}: [n] \to \mathbb Z$ is a surjection, and so there are at most $n$ equivalence classes. Because each equivalence class $\text{cl}(i)$ trivially contains $i$ for $0\leq i < n$, we have that there are at least $n$ equivalence classes. Therefore, there are exactly $n$ equivalence classes.
\end{enumerate}
\end{frame}
\begin{frame}
\frametitle{1.1 - Set Theory}
\small
\begin{enumerate}
	\item[(13)] It's clear that being in the same mutually disjoint subset $A_\alpha$ is a valid equivalence relation. This follows from the demonstrable reflexivity, symmetry and transitivity. Moreover, the equivalence classes must be the distinct $A_\alpha$'s because that is how we defined our equivalence.
\end{enumerate}
\end{frame}
\begin{frame}
\frametitle{1.2 - Mappings}
\small
As a matter of notation, I invoke $\sqrt{t}$ as meaning the \textit{positive square root} of $t$, so $\sqrt{t} \geq 0$.
\begin{enumerate}
	\item[(1a)] $\sigma$ is surjective, but not injective. Every $t\in T$ is mapped to the set $\sigma^{-1}(t) = \{-\sqrt{t}, \sqrt{t}\}$.
	\item[(1b)] $\sigma$ is both injective and surjective. Every $t\in T$ is mapped to its pre-image $\sigma^{-1}(t) = \sqrt{t}$.
	\item[(1c)] $\sigma$ is injective, but not surjective. Only the perfect squares $t\in T$ have pre-images of the form $\sigma^{-1}(t) = \sqrt{t}$.
	\item[(1d)] $\sigma$ is injective, but not surjective. Only the even integers $t\in T$ have pre-images of the form $\sigma^{-1}(t) = t/2$.
\end{enumerate}
\end{frame}
\begin{frame}
\frametitle{1.2 - Mappings}
\small
\begin{enumerate}
	\item[(2)] I define the injection $\alpha: S\times T\to T\times S$ as the ``swap'' map, $\alpha: (s, t)\mapsto(t,s)$. This map is provably an injection because
	\begin{equation*}
	(t, s) = (t', s')\quad\Leftrightarrow\quad(s, t) = (s', t')
	\end{equation*}
	So, this injection, $\alpha$, is evidence of the one-to-one correspondence between the sets.
	\item[(3a)] As in problem (2), the injection $\alpha: (S\times T)\times U\to S\times(T\times U)$ defined by $\alpha: ((s, t), u)\mapsto(s,(t,u))$ evidences a one-to-one correspondence between the sets.
	\item[(3b)] We define the injection $\alpha: (S\times T)\times U\to S\times T\times U$ via the map $\alpha: ((s,t),u)\mapsto(s,t,u)$, which evidences the one-to-one correspondence between the sets.
\end{enumerate}
\end{frame}
\begin{frame}
\frametitle{1.2 - Mappings}
\small
\begin{enumerate}
	\item[(4a)] If there's a one-to-one correspondence between $S$ and $T$, then there exists an injection $\sigma: S\to T$ satisfying $\sigma(s)=\sigma(s')\Rightarrow s = s'$. This injection induces an inverse injection, $\sigma^{-1}: T\to S$ defined by $\sigma^{-1}(t)=\left\{t\in T \mid \bigcup_{s\in S}\sigma(s) = t\right\}$. The injection $\sigma^{-1}$ evidences a one-to-one correspondence between $T$ and $S$.
	\item[(4b)] Suppose that the injections $\alpha: S\to T$ and $\beta: T\to U$ evidence the one-to-one correspondences between $S$ and $T$, and $T$ and $U$ respectively. Then we can define the composed injection $\sigma: S\to U$ via the map $\sigma: s\mapsto \beta(\alpha(s))$. Because $\beta(t)=\beta(t')\Rightarrow t=t'$ and $\alpha(s)=\alpha(s')\Rightarrow s=s'$, it follows that $\sigma(s)=\sigma(s')\Rightarrow s=s'$ and thus $\sigma$ is injective. This shows that there is a one-to-one correspondence between $S$ and $U$.
\end{enumerate}
\end{frame}
\begin{frame}
\frametitle{1.2 - Mappings}
\small
\begin{enumerate}
	\item[(5)] Because the identity automorphism $\iota: s\mapsto s$, it's clear that $\sigma\circ\iota: s\mapsto \sigma(s)$, which is identical to the map $\sigma: s\mapsto \sigma(s)$. An identical argument holds for $\iota\circ\sigma$, and so it holds that $\sigma=\sigma\circ\iota=\iota\circ\sigma$.
	\item[(6)] Because we know that $|S^*|>|S|$ for any set, it is impossible for any mapping of the $|S|$ elements of $S$ to cover the $|S^*|$ elements of $S^*$. Therefore, no mapping $S\to S^*$ will ever be surjective.
	\item[(7)] We can consider constructing an element $\sigma\in A(S)$ as a sequence of choosing unique images $\sigma(s_i)$ for each $s_i \in S$, $1\leq i \leq n$. $s_1$ could have any one of the $n$ elements of $S$ as its image. Each subsequent $s_i$ will have one fewer option for its image. This generates a set of $n!$ distinct automorphisms, and this set is $A(S)$.
\end{enumerate}
\end{frame}
\begin{frame}
\frametitle{1.2 - Mappings}
\small
\begin{enumerate}
	\item[(8a)] Suppose for the sake of argument that $\sigma: S\to S$ is surjective, but not injective. Specifically, suppose that there exists an $s^*\in S$ with more than one preimage, $\gamma = \sigma^{-1}(s^*)$. Then, it follows that $\sigma$ is still surjective if its restriction $\bar{\sigma}: S-\gamma\to S-{s^*}$ is surjective (note that we must remove $\gamma$ from our domain in order to enforce the constraint that an element of the domain cannot have two different images). However, the domain of $\bar{\sigma}$ is smaller than its codomain, which means that $\bar{\sigma}$ cannot be surjective because no element of the domain can map to two different images. Therefore, neither $\bar{\sigma}$ nor $\sigma$ itself can be surjective without also being injective.
\end{enumerate}
\end{frame}
\begin{frame}
\frametitle{1.2 - Mappings}
\small
\begin{enumerate}
	\item[(8b)] Using the reverse argument from problem (8a), we suppose for the sake of contradiction that $\sigma$ is not surjective. Because it \textit{is} injective, we have that every element $s\in S$ in the domain is mapped to a unique image $\sigma(s)\in S$. Because we assume it is not surjective, we assume that there is an $s^*\in S$ which does not have a preimage. So, $\sigma$ maps the $|S|$ elements of $S$ to the $|S| - 1$ elements of $S - \{s^*\}$. By the pigeonhole principle, $\sigma$ must map two distinct inputs to the same image, and so $\sigma$ cannot be injective. We have a contradiction, so we prove the result
	\begin{equation*}	
	\sigma\text{ is injective}\quad\Leftrightarrow\quad\sigma\text{ is surjective}
	\end{equation*}
\end{enumerate}
\end{frame}
\begin{frame}
\frametitle{1.2 - Mappings}
\small
\begin{enumerate}
	\item[(8c)] \quad Consider the mapping $\sigma: \mathbb Z\to\mathbb Z$ via the assignment $n\mapsto \lfloor n / 2\rfloor$. Clearly, $\sigma$ is surjective because every integer can be doubled to result in one of its preimages. However, it is not injective because $n$ has the multiple preimages $\{2n, 2n+1\}$. So, $\sigma$ is an infinite counter-example to (8a).
	
	\quad On the other hand, the mapping $\sigma: \mathbb Z \to \mathbb Z$ via the assignment $n\mapsto 2n$ is clearly injective because every integer can be doubled to result in a unique image. However, it is not surjective because the odd integers have no preimages (they are not the result of any integer being doubled). This evidences an infinite counter-example to (8b).
\end{enumerate}
\end{frame}
\begin{frame}
\frametitle{1.2 - Mappings}
\small
\begin{enumerate}
	\item[(9a)] Consider the maps 
	\begin{align*}	
	\sigma: [2]\to [2]\quad &\quad \sigma: i\mapsto 1 \\
	\tau: [2]\to [1]\quad &\quad \tau: 1\mapsto 1
	\end{align*}
	Clearly, $\sigma$ is not surjective because $2$ has no preimage. However, $\sigma\circ\tau$ is surjective because $1$ does have a preimage. So, the converse of the lemma is false.
	\item[(9b)] Consider the maps
	\begin{align*}
	\sigma: [1]\to [2]\quad &\quad \sigma: 1\mapsto 1 \\
	\tau: [2]\to [2]\quad &\quad \tau: i\mapsto 1
	\end{align*}
	Clearly, $\tau$ is not injective because both $1$ and $2$ are mapped to the same image, $1$. However, $\sigma\circ\tau$ is injective because $1$ has only one preimage. So, the converse of the lemma is false.
\end{enumerate}
\end{frame}
\begin{frame}
\frametitle{1.2 - Mappings}
\small
\begin{enumerate}
	\item[(10)] We define the map $\kappa: \mathbb Z \to \mathbb Q$ via the assignment based on prime factorization of each integer:
	\begin{equation*}
	p_1^{r_1} p_2^{r_2}\cdots p_n^{r_n}\quad\mapsto\quad \frac{p_1^{r_1}}{p_2^{r_2}\cdots p_n^{r_n}}
	\end{equation*}
	It's clear that the numerator and denominator are coprime, and so the image of each integer is a valid rational. Moreover, every rational has at most one preimage. If two images $(p/q)$ and $(p'/q')$ are equal, then it means that their components $p=p'$ and $q=q'$, and so their preimages are equal, $pq = p'q'$. This shows that $\kappa$ is an injection and so there is a one-to-one correspondence between the integers and the rationals.
\end{enumerate}
\end{frame}
\begin{frame}
\frametitle{1.2 - Mappings}
\small
\begin{enumerate}
	\item[(11a)] Trivially, because $\sigma$ is a mapping into $T$, and $\sigma(A) = \sigma_A(A)$, $\sigma_A$ is thus a mapping $A\to T$.
	\item[(11b)] If we assume that $\sigma$ is injective, then we know that every element of $\sigma(S)$ is the image of exactly one $s\in S$. Because $\sigma(A)\subset\sigma(S)$, we have that every element of $\sigma(A)$ is the image of exactly one $a\in A$. Because $\sigma(A) = \sigma_A(A)$, the same property is true of $\sigma_A$'s domain. Therefore, $\sigma_A$ is injective. 
	\item[(11c)] If the set of elements, $\bar{T}\subset T$, in $T$ which have more than one preimage is disjoint from $\sigma(A)$ such that $\sigma(A)\cap\bar{T}=\emptyset$, then $\sigma$ will still have the property of mapping each $a\in A$ to a unique image $\sigma(a)\in\sigma(A)$. Therefore, $\sigma_A$ can still be injective, even if $\sigma$ as a whole is not.
\end{enumerate}
\end{frame}
\begin{frame}
\frametitle{1.2 - Mappings}
\small
\begin{enumerate}
	\item[(12)] First, we recognize that $\sigma(A)\subset A$, so $\sigma\circ\sigma(A)\subset A$. On the other hand, $\sigma_A(A) = \sigma(A) \subset A$, and likewise for $\sigma_A\circ\sigma_A(A)$. Because of this equality, we have that $\sigma\circ\sigma(A) = \sigma_A\circ\sigma_A(A)$, and so the domain of both $(\sigma\circ\sigma)_A$ and $\sigma_A\circ\sigma_A$ are mapped to the same images, and so the functions are the same.
	\item[(13a)] Consider the proper subset $5\mathbb Z \subset \mathbb Z$. The injection $\sigma: x\mapsto 5x$ evidences a one-to-one correspondence between the sets, and so $\mathbb Z$ is infinite.
	\item[(13b)] Consider the proper subset $\mathbb R_{>0} \subset \mathbb R$. The injection $\sigma: x\mapsto e^x$ evidences a one-to-one correspondence between the sets, and so $\mathbb R$ is infinite.
	\item[(13c)] Because $A$ is infinite, it has a subset $\bar{A}\subset A\subset S$ with which it has a one-to-one correspondence. This means that $S$ also has the same one-to-one correspondence with $\bar{A}$ and so $S$ is also infinite.
\end{enumerate}
\end{frame}
\begin{frame}
\frametitle{1.2 - Mappings}
\small
\begin{enumerate}
	\item[(14)] Let ``$S\to\mathbb Z$'' denote that there is a one-to-one correspondence, called $\alpha$, between $S$ and $\mathbb Z$. Because $\mathbb Z\to\mathbb Q$ via the mapping $\kappa$ from problem (10) and the transitivity proved in problem (4b), we have $S\to\mathbb Q$. Moreover, via the injection $f: \mathbb Q\to \mathbb Z\times\mathbb Z$ with map $p/q \mapsto (p, q)$, we establish $\mathbb Q \to\mathbb Z\times\mathbb Z$. Finally, by inverting and pairng $\alpha$, we can inject $A^{-1}: \mathbb Z\times \mathbb Z\to S\times S$ via the assignment $A^{-1}: (p,q)\mapsto (\alpha^{-1}(p), \alpha^{-1}(q))$. This injection completes our transitive chain by implying $S\to S\times S$, as desired.
\end{enumerate}
\end{frame}
\begin{frame}
\frametitle{1.2 - Mappings}
\small
\begin{enumerate}
	\item[(15)] \quad First, we show that there exists a surjective map $\sigma: U\to S$. This follows from the existence of surjective maps $\alpha: U\to T$ and $\beta: T\to S$, which compose to give $\sigma = \alpha\circ\beta$. 
	
	\quad Second, we consider a hypothetical surjective map $\gamma: S\to U$. If such a map existed, its inputs could first be mapped into $T$, then mapped surjectively onto $U$. This violates our assumption that there does not exist a surjective map between $T$ and $U$. So, there cannot exist a surjective map from $S\to U$. Therefore, $S<U$.
\end{enumerate}
\end{frame}
\begin{frame}
\frametitle{1.2 - Mappings}
\small
\begin{enumerate}
	\item[(16)] \quad If we start by supposing that $m<n$, then we can easily find a surjective map $\sigma: T\to S$ by the pigeonhole principle. Moreover, we cannot form a surjective map $\gamma: S\to T$. If we assign every $s\in S$ in the domain to a unique image $t\in T$, then there will always be at least one element $t^*\in T$ with no preimage in $S$. Therefore, there can be no surjective map from $S\to T$. So, $S<T$.
	
	\quad Conversely, if we start by supposing that $S<T$, then it is evident that there cannot exist a surjective map $S\to T$. If $m\geq n$, then we can easily find a surjective map onto $T$ by the pigeonhole principle. Therefore, this enforces that $m<n$.
	
	\quad Both directions prove that $S<T$ for finite sets $S,T$ equivalently means $m<n$.
\end{enumerate}
\end{frame}
\end{document}