\documentclass{beamer}
\usepackage{amsmath}
\usefonttheme{serif}
\title{Topics in Algebra, Chapter 2 Solutions}
\author{David Sillman}
\date{2022}


\begin{document}

\frame{\titlepage}

\begin{frame}
\frametitle{2.3 - Groups}
\small
\begin{enumerate}
	\item[(1a)] $G = (\mathbb Z, -)$ does not form a group because it does not satisfy associativity:
	\begin{equation*}
	a-(b-c) = a-b+c\qquad (a-b)-c = a+b-c
	\end{equation*}
	\item[(1b)] $G = (\mathbb Z, \cdot)$ does not form a group because it does not contain multiplicative inverses for all integers,
	\begin{equation*}
	ab = 1\quad\Leftrightarrow\quad a=b=\pm 1
	\end{equation*}
	\item[(1c)] $G = (\{a_i\}, \cdot)$ forms a group because it satisfies (1) closure, (2) associativity, (3) identity and (4) inverse:
	\begin{gather}
	a_i\cdot a_j = a_{i+j}\quad\text{s.t.}\quad 0\leq i+j\leq 6\\
	a_i\cdot(a_j\cdot a_k) = a_i \cdot a_{j+k} = a_{i+j+k} = a_{i+j}\cdot a_{k}= (a_i\cdot a_j)\cdot a_k\\
	a_i\cdot a_0 = a_{i+0} = a_i = a_{0+i} = a_0\cdot a_i\\
	a_i\cdot a_{7-i} = a_{i+7-i} = a_0 = a_{7-i+i} = a_{7-i}\cdot a_i
	\end{gather}
\end{enumerate}
\end{frame}
\begin{frame}
\frametitle{2.3 - Groups}
\small
\begin{enumerate}
	\setcounter{equation}{0}
	\item[(1d)] $G = (\mathbb Q^{\text{odd}}, +)$ forms a group because it satisfies (1) closure, (2) associativity, (3) identity and (4) inverse:
	\begin{gather}
	\frac{a}{b},\frac{a'}{b'}\in\mathbb Q^{\text{odd}}\quad\Rightarrow\quad\frac{a}{b}+\frac{a'}{b'} = \frac{ab' + a'b}{bb'}\in\mathbb Q^{\text{odd}} \\
	\frac{a}{b} + \left(\frac{a'}{b'} + \frac{a''}{b''}\right) = \left(\frac{a}{b}+\frac{a'}{b'}\right)+\frac{a''}{b''} \\
	\frac{a}{b} + \frac{0}{1} = \frac{0}{1} + \frac{a}{b} = \frac{a}{b} \\
	\frac{a}{b} + \frac{-a}{b} = \frac{-a}{b} + \frac{a}{b} = \frac{0}{b} = \frac{0}{1}
	\end{gather}
\end{enumerate}
\end{frame}
\begin{frame}
\frametitle{2.3 - Groups}
\small
\begin{enumerate}
	\item[(2)] If $G$ is abelian, then we can expand, commute and regroup terms:
	\begin{equation*}
	(a\cdot b)^n = (a\cdot b)\cdots(a\cdot b) = a\cdots a\cdot b\cdots b = a^n\cdot b^n
	\end{equation*}
	\item[(3)] Expanding and disassociating the product, we get $(a\cdot b)^2 = (a\cdot b)(a\cdot b) = a\cdot b\cdot a\cdot b$.  For the equality $a\cdot b\cdot a\cdot b = a\cdot a\cdot b\cdot b = a^2\cdot b^2$ to hold, we can left-multiply by $a^{-1}$ and right-multiply by $b^{-1}$:
	\begin{gather*}
	a^{-1}\cdot a\cdot b\cdot a\cdot b\cdot b^{-1} = a^{-1}\cdot a\cdot a\cdot b\cdot b\cdot b^{-1} \\
	\Downarrow \\
	b\cdot a = a\cdot b
	\end{gather*}
	Which shows that every pair $a,b\in G$ must commute, which means $G$ is abelian.
\end{enumerate}
\end{frame}
\begin{frame}
\frametitle{2.3 - Groups}
\small
\begin{enumerate}
	\item[(4)] Let $i\in\mathbb Z$ be the least of the 3 consecutive integers which satisfy the equation. If $i=0$, then the result of problem (3) trivially proves the result. Otherwise, if $i>0$, then let $x=(a\cdot b)^i = a^i\cdot b^i$. Then, we have $x\cdot a\cdot b = a\cdot x\cdot b$. We cancel the rightmost $b$ on both sides, giving $a\cdot x = x\cdot a$, which means we can move $a$ from one side of $x$ to the other. This implies that
	\begin{gather*}
	a^2\cdot x\cdot b^2 = x\cdot (a\cdot b)^2 \\
	a\cdot x\cdot a\cdot b^2 = x\cdot (a\cdot b)^2 \\
	x\cdot a^2\cdot b^2 = x\cdot (a\cdot b)^2 \\
	a^2\cdot b^2 = (a\cdot b)^2
	\end{gather*}
	Which then gives us the statement from problem (3), which trivially shows that $a\cdot b = b\cdot a$ for any pair $a,b\in G$. Therefore, $G$ is abelian.
\end{enumerate}
\end{frame}
\begin{frame}
\frametitle{2.3 - Groups}
\small
\begin{enumerate}
	\item[(5)] If we don't have $a^{i+2}\cdot b^{i+2} = (a\cdot b)^{i+2}$, then we can still show that $a\cdot x = x\cdot a$ with $x=a^i\cdot b^i = (a\cdot b)^i$. Canceling out $x$ now only gives us the trivial tautology,
	\begin{gather*}
	a\cdot x\cdot b = x\cdot a\cdot b \\
	x\cdot a\cdot b = x\cdot a\cdot b \\
	a\cdot b = a\cdot b
	\end{gather*}
	Which does not necessitate nor imply that $a\cdot b = b\cdot a$, and so we don't guarantee that $G$ is abelian.
\end{enumerate}
\end{frame}
\begin{frame}
\frametitle{2.3 - Groups}
\small
\begin{enumerate}
	\item[(6)] In $S_3$, we have two elements (transpositions) $a =(1\;2)$ and $b = (2\;3)$ satisfying $a^2 = e$ and $b^2 = e$. They multiply to give $a\cdot b = (1\;2)\cdot (2\;3) = (1\;2\;3)$. Squaring each and multiplying them gives:
	\begin{equation*}
	a^2\cdot b^2 = e\cdot e = e
	\end{equation*}
	Whereas multiplying them, then squaring, gives:
	\begin{gather*}
	(a\cdot b)^2 = (1\;2\;3)^2 = (1\;3\;2) \neq e
	\end{gather*}
	\item[(7)] The elements of order $2$ in $S_3$ are the \textit{transpositions} and identity, which are
	\begin{equation*}
	(1\;2)^2 = (2\;3)^2 = (1\;3)^2 = e^2 = e
	\end{equation*}
	The elements of order $3$ in $S_3$ are the \textit{shifts} and identity, which are
	\begin{equation*}
	(1\;2\;3)^3 = (1\;3\;2)^3 = e^3 = e
	\end{equation*}
\end{enumerate}
\end{frame}
\begin{frame}
\frametitle{2.3 - Groups}
\small
\begin{enumerate}
	\item[(8)] \quad First, I argue that $a^i= a^j$ for two positive integers, $0<i<j$. This comes from the fact that $G$ has a finite number of elements, $|G|\in\mathbb N$, and so for multiplication by $a$ to be closed, $a^i$ must be one of the $|G|$ elements for all $i\in\mathbb Z$. By the pigeonhole principle, $a^{|G|}$ must be in $\{e, a, a^2, \ldots, a^{|G|-1}\}$. Therefore, we have $a^i = a^{|G|}$ for some $0\leq i <|G|$.
	
	\quad With this equality, we can left-multiply through by $(a^i)^{-1}$, which gives $a^{|G| - i} = e$, where $|G| - i$ is a positive integer, which proves the result.
\end{enumerate}
\end{frame}
\begin{frame}
\frametitle{2.3 - Groups}
\small
\begin{enumerate}
	\item[(9a)] If $G$ has 3 elements, then either it contains:
	\begin{align*}
	G &= \{e, a, a^{-1}\}\qquad\text{with } a\neq a^{-1}\\
	G &= \{e, a, b\}\qquad\text{with } a = a^{-1}\text{ and }b=b^{-1}
	\end{align*}
	The former is trivially abelian. In the latter, we must try to assign the element $a\cdot b$. If $a\cdot b = a$ or $a\cdot b = b$, then this implies $b=e$ or $a=e$, respectively. To avoid contradiction, we assign $a\cdot b = e$. Inverting both sides gives $(a\cdot b)^{-1} = b^{-1}\cdot a^{-1} = b\cdot a = e$, which means $a\cdot b = b\cdot a$, so $G$ is abelian.
\end{enumerate}
\end{frame}
\begin{frame}
\frametitle{2.3 - Groups}
\small
\begin{enumerate}
	\item[(9b)] If $G$ has 4 elements, then we can consider the act of left-multiplying by any non-identity element $a\in G$ as a map, $l: G\to G$ which maps $e\mapsto a$ and $a^{-1}\mapsto e$. This defines assignments for two of the four elements of $G$. Neither of the two remaining elements, $x,y\in G$ can be mapped to themselves, as this would imply $a=e$. All of the above argument holds for the right-multiplication map, $r: G\to G$, and so the act of left- and right-multiplication in $G$ are identical, meaning $a\cdot b = b\cdot a$. So, $G$ is commutative.
\end{enumerate}
\end{frame}
\begin{frame}
\frametitle{2.3 - Groups}
\small
\begin{enumerate}
	\item[(9c)] Suppose $G$ has $5$ elements. If we choose a non-identity element $a\in G$ at random, there are two cases:
	
	\quad (i) $a^5 = e$: in this case, every element is expressible as $a^i$, and so $a^i\cdot a^j = a^j\cdot a^i$ by associativity, so $G$ is trivially abelian.
	
	\quad (ii) $a^i = e$ for $1 < i < 5$: in this case, for any other element $b\in G$, we need $a\cdot b, b\cdot a, a\cdot b\cdot a\in G$. If we assume non-abelian properties, we require that all of these products be distinct, and so $G$ must contain at least $G =\{e, a, b, a\cdot b, b\cdot a, a\cdot b\cdot a\}$, which is more than 5 elements, so we have a contradiction.

	\quad Therefore, any 5-element group must be abelian.
\end{enumerate}
\end{frame}
\begin{frame}
\frametitle{2.3 - Groups}
\small
\begin{enumerate}
	\item[(10)] Indeed, if every element $a\in G$ satisfies $a^{-1} = a$, then $a\cdot b\in G$ satisfies $a\cdot b = (a\cdot b)^{-1} = b^{-1}\cdot a^{-1} = b\cdot a$, and so $G$ is abelian.
	\item[(11)] Let $|G| = 2n$ and $\phi: G\to G$ be the map which sends $a\mapsto a^{-1}$, which is a bijection satisfying $\phi^{2}(a) = a$. Trivially, $\phi(e) = e$. This means there remain $2n - 1$ unassigned elements in the domain and range. Each new assignment $\phi(a) = a'$ assigns $\phi(a') = a$, and so assignments can only be made in pairs. Ultimately, when there remains only 1 unassigned element $x\in G$, we must assign $\phi(x) = x$ for that $x$. Therefore, any even-order group $G$ must have some $a\in G$ satisfying $a^2 = e$.
\end{enumerate}
\end{frame}
\begin{frame}
\frametitle{2.3 - Groups}
\small
\begin{enumerate}
	\item[(12)] Taking the expression $a\cdot e = a$ and right-multiplying by $y(a)$ gives us $a\cdot e\cdot y(a) = a\cdot y(a) = e$, which shows $e\cdot a = a\cdot e$. Substitution on the left gives $a\cdot y(a)\cdot a = a$, which associates to give $a\cdot(y(a)\cdot a) = a$, so $y(a)\cdot a = e$. Therefore, both inverses and the identity commute. So, $G$ is a group.
	\item[(13)] Consider the set $G = \mathbb Z$ closed under the associative product $a\cdot b = a$. Trivially, we have $a\cdot 1 = a$, and the $y(a)$ which satisfies $y(a)\cdot a = 1$ will be $y(a) = 1$ for all $a$. Clearly, this satisfies all conditions without forming a group, as neither identity nor inverses are unique.
\end{enumerate}
\end{frame}
\end{document}